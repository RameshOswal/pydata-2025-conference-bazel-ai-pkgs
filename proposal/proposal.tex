\documentclass[11pt,a4paper]{article}
\usepackage[margin=1in]{geometry}
\usepackage{graphicx}
\usepackage{hyperref}
\usepackage{enumitem}
\usepackage{titlesec}
\usepackage{xcolor}

% Formatting
\hypersetup{
    colorlinks=true,
    linkcolor=blue,
    urlcolor=blue,
    citecolor=blue
}

\titleformat{\section}
  {\normalfont\Large\bfseries\color{blue!60!black}}{\thesection}{1em}{}

\titleformat{\subsection}
  {\normalfont\large\bfseries}{\thesubsection}{1em}{}

\setlength{\parindent}{0pt}
\setlength{\parskip}{0.5em}
% https://www.overleaf.com/project/690f034af7653300907b3a2e
\begin{document}

% Title Page
\begin{center}
    {\Huge\bfseries Building Bazel Packages for AI/ML:\\SciPy, PyTorch, and Beyond}
    
    \vspace{1.5cm}
    
    {\Large Conference Proposal}
    
    \vspace{0.5cm}
    
    {\large PyData Seattle 2025}
    
    \vspace{2cm}
    
    {\Large\textbf{Speakers:}}
    
    \vspace{0.5cm}
    
    {\large Ramesh Oswal}
    
    {\large Jiten Oswal}
    
    \vspace{2cm}
    
    {\large \today}
\end{center}

\newpage

% Table of Contents
\tableofcontents
\newpage

\section{Proposal Overview}

\subsection{Title}
Building Bazel Packages for AI/ML: SciPy, PyTorch, and Beyond

\subsection{Speakers}
\begin{itemize}[leftmargin=2em]
    \item \textbf{Ramesh Oswal}
    \item \textbf{Jiten Oswal}
\end{itemize}

\subsection{Session Duration}
280 minutes (4 hours 40 minutes)

\subsection{Target Audience}
\begin{itemize}[leftmargin=2em]
    \item Software engineers working with AI/ML infrastructure
    \item DevOps and MLOps practitioners
    \item Data scientists interested in reproducible builds
    \item Build system engineers
    \item Researchers managing complex dependency chains
\end{itemize}

\section{Abstract}

AI/ML workloads depend heavily on complex software stacks, including numerical computing libraries (SciPy, NumPy), deep learning frameworks (PyTorch, TensorFlow), and specialized toolchains (CUDA, cuDNN). However, integrating these dependencies into Bazel-based workflows remains challenging due to compatibility issues, dependency resolution, and performance optimization. 

This session explores the process of creating and maintaining Bazel packages for key AI/ML libraries, ensuring reproducibility, performance, and ease of use for researchers and engineers.

\section{Session Outline}

\subsection{Introduction to Bazel for AI/ML (20 minutes)}

This section introduces participants to Bazel and its relevance for AI/ML workflows:

\begin{itemize}[leftmargin=2em]
    \item Overview of Bazel build system fundamentals
    \item Why Bazel matters for AI/ML workloads
    \item Benefits of Bazel for dependency management
    \item Reproducibility and hermetic builds
    \item Cross-platform compatibility considerations
    \item Comparison with traditional Python packaging (pip, conda)
\end{itemize}

\textbf{Learning Outcomes:}
\begin{itemize}[leftmargin=2em]
    \item Understand Bazel's core concepts and terminology
    \item Recognize the advantages of Bazel for AI/ML projects
    \item Identify use cases where Bazel adds value
\end{itemize}

\subsection{Challenges in AI/ML Bazel Packaging (30 minutes)}

An in-depth exploration of the unique challenges encountered when packaging AI/ML libraries with Bazel:

\begin{itemize}[leftmargin=2em]
    \item \textbf{Transitive Dependencies:} Managing complex dependency graphs
    \item \textbf{Build System Differences:} Bridging CMake, setuptools, and Bazel
    \item \textbf{GPU Acceleration:} Integrating CUDA, cuDNN, and hardware-specific toolchains
    \item \textbf{Binary Compatibility:} ABI compatibility across different platforms
    \item \textbf{Large Binary Sizes:} Handling large ML model files and libraries
    \item \textbf{Version Conflicts:} Resolving conflicts between library versions
    \item \textbf{Native Extensions:} Compiling C/C++ extensions for Python libraries
\end{itemize}

\textbf{Learning Outcomes:}
\begin{itemize}[leftmargin=2em]
    \item Identify common pitfalls in AI/ML library packaging
    \item Understand dependency resolution complexities
    \item Recognize GPU toolchain integration challenges
\end{itemize}

\subsection{Strategies for Packaging (30 minutes)}

Practical strategies and techniques for successfully packaging AI/ML libraries:

\begin{itemize}[leftmargin=2em]
    \item \textbf{SciPy Stack:} Building NumPy, SciPy, and related numerical libraries
    \item \textbf{PyTorch:} Creating Bazel rules for PyTorch with CUDA support
    \item \textbf{TensorFlow:} Leveraging existing Bazel infrastructure
    \item \textbf{Dependency Management:} Strategies for managing transitive dependencies
    \item \textbf{Performance Optimization:} Ensuring optimal build and runtime performance
    \item \textbf{Cross-Platform Support:} Building for Linux, macOS, and Windows
    \item \textbf{Testing and Validation:} Ensuring package correctness and functionality
\end{itemize}

\textbf{Learning Outcomes:}
\begin{itemize}[leftmargin=2em]
    \item Apply proven packaging strategies to common libraries
    \item Optimize builds for performance and compatibility
    \item Implement effective testing strategies
\end{itemize}

\subsection{Best Practices for Distribution and Maintenance (20 minutes)}

Guidelines for maintaining and distributing Bazel packages in production environments:

\begin{itemize}[leftmargin=2em]
    \item \textbf{Version Management:} Semantic versioning and release strategies
    \item \textbf{Repository Structure:} Organizing Bazel workspaces effectively
    \item \textbf{Distribution Channels:} Using Bazel Central Registry and private registries
    \item \textbf{Documentation:} Creating user-friendly package documentation
    \item \textbf{CI/CD Integration:} Automating build and test pipelines
    \item \textbf{Community Engagement:} Contributing to open-source Bazel rules
    \item \textbf{Monitoring and Updates:} Keeping packages up-to-date with upstream changes
\end{itemize}

\textbf{Learning Outcomes:}
\begin{itemize}[leftmargin=2em]
    \item Establish sustainable maintenance workflows
    \item Implement effective distribution strategies
    \item Integrate packages into CI/CD pipelines
\end{itemize}

\subsection{Hands-on Demo (144 minutes)}

An extensive practical demonstration where participants follow along:

\begin{itemize}[leftmargin=2em]
    \item \textbf{Environment Setup:} Configuring Bazel for AI/ML development
    \item \textbf{Building SciPy:} Step-by-step package creation for SciPy
    \item \textbf{Integrating PyTorch:} Adding PyTorch with GPU support
    \item \textbf{Creating Custom Rules:} Writing Bazel rules for custom libraries
    \item \textbf{Dependency Resolution:} Handling complex dependency chains
    \item \textbf{Testing and Validation:} Running tests to verify package functionality
    \item \textbf{Performance Benchmarking:} Comparing build times and runtime performance
    \item \textbf{Troubleshooting:} Common issues and debugging techniques
\end{itemize}

\textbf{Learning Outcomes:}
\begin{itemize}[leftmargin=2em]
    \item Build complete Bazel packages for real AI/ML libraries
    \item Debug and troubleshoot packaging issues
    \item Apply learned techniques to custom projects
\end{itemize}

\subsection{Q\&A and Open Discussion (36 minutes)}

An interactive session for:

\begin{itemize}[leftmargin=2em]
    \item Addressing participant questions
    \item Discussing specific use cases and challenges
    \item Sharing experiences and best practices
    \item Exploring future directions for Bazel in AI/ML
    \item Networking and community building
\end{itemize}

\section{Key Takeaways}

Participants will leave this session with:

\begin{enumerate}[leftmargin=2em]
    \item \textbf{Practical Knowledge:} Hands-on experience building Bazel packages for AI/ML libraries
    \item \textbf{Problem-Solving Skills:} Techniques to overcome common packaging challenges
    \item \textbf{Best Practices:} Industry-standard approaches to package maintenance and distribution
    \item \textbf{Reproducible Workflows:} Ability to create hermetic, reproducible build environments
    \item \textbf{Community Resources:} Knowledge of community tools, resources, and support channels
\end{enumerate}

\section{Prerequisites}

To get the most out of this session, participants should have:

\begin{itemize}[leftmargin=2em]
    \item Basic understanding of build systems and dependency management
    \item Familiarity with Python and AI/ML libraries
    \item Basic command-line proficiency
    \item (Optional) Prior exposure to Bazel or similar build systems
    \item Laptop with Docker or Bazel installed (for hands-on portion)
\end{itemize}

\section{Materials Provided}

\begin{itemize}[leftmargin=2em]
    \item Complete Bazel workspace examples
    \item Sample BUILD files for SciPy, PyTorch, and other libraries
    \item Documentation and reference guides
    \item Troubleshooting checklists
    \item Links to additional resources and community support
\end{itemize}

\section{Speaker Information}

\subsection{Ramesh Oswal}

Ramesh Oswal is an experienced software engineer specializing in build systems and AI/ML infrastructure. With extensive experience in Bazel and large-scale dependency management, Ramesh has contributed to numerous open-source projects focused on reproducible ML workflows.

\subsection{Jiten Oswal}

Jiten Oswal brings expertise in DevOps, MLOps, and infrastructure automation. His work focuses on creating efficient, scalable build pipelines for AI/ML applications, with a particular emphasis on cross-platform compatibility and performance optimization.

\section{Why This Session Matters}

As AI/ML workloads become increasingly complex and production-critical, the need for reliable, reproducible build systems has never been greater. Traditional Python packaging approaches often fall short when dealing with:

\begin{itemize}[leftmargin=2em]
    \item Complex native dependencies (CUDA, cuDNN, MKL)
    \item Large-scale monorepo structures
    \item Strict reproducibility requirements
    \item Cross-platform deployment needs
    \item High-performance computing environments
\end{itemize}

This session addresses these challenges head-on, providing practical solutions that teams can implement immediately. By the end of the session, participants will have the skills and confidence to manage their own AI/ML dependencies using Bazel, leading to more reliable and maintainable systems.

\section{Expected Outcomes}

After attending this session, participants will be able to:

\begin{enumerate}[leftmargin=2em]
    \item Create Bazel packages for popular AI/ML libraries
    \item Integrate GPU acceleration toolchains into Bazel builds
    \item Manage complex dependency graphs effectively
    \item Set up reproducible build environments for AI/ML projects
    \item Troubleshoot common packaging issues
    \item Contribute to the Bazel AI/ML ecosystem
\end{enumerate}

\section{Contact Information}

For questions or additional information about this proposal, please contact the speakers through the GitHub repository:

\url{https://github.com/RameshOswal/pydata-2025-conference-bazel-ai-pkgs}

\vspace{2cm}

\begin{center}
\line(1,0){250}

\textit{Thank you for considering this proposal for PyData Seattle 2025!}
\end{center}

\end{document}
